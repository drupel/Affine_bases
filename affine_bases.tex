\documentclass{amsart}
\usepackage[margin=1in]{geometry}
\usepackage{amsmath,amssymb,latexsym}

\newtheorem{theorem}{Theorem}
\newtheorem{lemma}[theorem]{Lemma}
\newtheorem{definition}[theorem]{Definition}
\newtheorem{proposition}[theorem]{Proposition}

\newcommand{\bfd}{\mathbf{d}}
\newcommand{\bfg}{\mathbf{g}}
\newcommand{\diag}{\operatorname{diag}}
\newcommand{\ol}[1]{\overline{#1}}
\newcommand{\NN}{\mathbb{N}}
\newcommand{\TT}{\mathbb{T}}
\newcommand{\ZZ}{\mathbb{Z}}

\title{All bases for acyclic affine cluster algebras consist of generalized minors}

\author{Dylan Rupel}
\author{Salvatore Stella}

\begin{document}
  \begin{abstract}
    We prove what the title says.
    Along the way we show that all cluster monomials for \emph{every} acyclic cluster algebra are computable as generalized minors.
  \end{abstract}
  \maketitle

  \section{Affine Roots Systems}
    \begin{theorem}
      Characterizations of $\delta$
    \end{theorem}

  \section{Cluster Algebras}
    Let $B_0=(b_{ij})$ denote an $n\times n$ skew-symmetrizable matrix with skew-symmetrizing matrix $D=\diag(d_1,\ldots,d_n)$, i.e. $DB_0$ is skew-symmetric.
    Define the \emph{mutation} of $B_0$ in direction $k$ by $\mu_k B_0:=(b'_{ij})$ for
    \[b'_{ij}=\begin{cases} -b_{ij} & \text{if $i=k$ or $j=k$;}\\ b_{ij}+b_{ik}[b_{kj}]_++[-b_{ik}]_+b_{kj} & \text{otherwise;}\end{cases}\]
    where we used the notation $[b]_+=\max(b,0)$.
    An easy calculation shows that $D\mu_k B_0$ is again skew-symmetric and that $\mu_k \mu_k B_0 = B_0$.

    To record sequences of matrix mutations, we introduce the rooted $n$-regular tree $\TT_n$ with root vertex $t_0$ and with the edges adjacent to each vertex labeled by the set $\{1,2,\ldots,n\}$.
    We associate a skew-symmetrizable matrix $B_t$ to each vertex $t\in\TT_n$ as follows:
    \begin{itemize}
      \item $B_{t_0}=B_0$;
      \item if $t$ and $t'$ are joined by an edge labeled $k$, then $B_{t'}=\mu_k B_t$.
    \end{itemize}

    Alongside the matrix mutations, we consider for each $B_t=(b^{(t)}_{ij})$ transformations $\phi_{t,k}$ of the integer lattice $\ZZ^n$ given on $g=(g_i)$ by $\phi_{t,k}(g)=(g'_i)$ for
    \[g'_i=\begin{cases} -g_k & \text{if $i=k$;}\\ g_i+b^{(t)}_{ik}[g_k]_++[-b^{(t)}_{ik}]_+g_k & \text{otherwise.}\end{cases}\]
    Given a pair of vertices $t,t'\in\TT_n$ joined together by mutations in directions $k_1,k_2,\ldots,k_r$ through vertices $t=:t_1,t_2,\ldots,t_{r+1}:=t'$ we define the composite transformation of $\ZZ^n$ given by
    \[\phi_{t',t}:= \phi_{t_r,k_r}\cdots \phi_{t_2,k_2}\phi_{t_1,k_1}.\]

    The matrices $B_t$ also determine a particular dominance order on the lattice $\ZZ^n$.
    \begin{definition}
      Given $t\in\TT_n$ and $g,g'\in\ZZ^n$, write $g'\prec_t g$ if there exists $h\in\NN^n$ so that $g'=g+B_t h$.
    \end{definition}

    \begin{theorem}
      $\bfg$-vectors from $\bfd$-vectors
    \end{theorem}

  \section{Level Zero Minors}
    \subsection{Level Zero Representations}
    \begin{definition}
      Chari-Pressley level zero representations
    \end{definition}

    \begin{theorem}
      explicit type-by-type analysis of how to construct the minor corresponding to the generic basis element $x_{\delta}^n$
    \end{theorem}

    \begin{theorem}
      \mbox{}
      \begin{enumerate}
        \item only a single step is possible in the $\theta$ direction for finite-type representations found above
        \item this can also be achieved uniquely by walking along the chosen coxeter element
      \end{enumerate}
    \end{theorem}

  \section{Coxeter Double Bruhat Cells}
    \begin{theorem}
      cluster structure on $G^{c,c^{-1}}$
    \end{theorem}

    \begin{proposition}
      \cite[Prop. 2.2]{rsw19}
      Any double reduced word produces a factorization of a generic element of $G^{c,c^{-1}}$.
    \end{proposition}
    
    Define the following elements of $G$:
    \begin{align}
      \label{eq:generic factorizations}
      g_k:=&x_{\ol{1}}(r_{\ol{1}}^{(k)}) x_{\ol{2}}(r_{\ol{2}}^{(k)}) \cdots x_{\ol{n-k}}(r_{\ol{n-k}}^{(k)}) x_n(r_n^{(k)}) x_{n-1}(r_{n-1}^{(k)}) \cdots x_{n-k+1}(r_{n-k+1}^{(k)})\times\\
      \nonumber & \quad \times x_{\ol{n-k+1}}(r_{\ol{n-k+1}}^{(k)}) \cdots x_{\ol{n-1}}(r_{\ol{n-1}}^{(k)}) x_{\ol{n}}(r_{\ol{n}}^{(k)}) x_{n-k}(r_{n-k}^{(k)}) \cdots x_2(r_2^{(k)}) x_1(r_1^{(k)}) h_k.
    \end{align}
    We use these elements to understand the minors appearing along the sequence of seeds
    \[t_0 \stackrel{n}{\longleftrightarrow} t_{-1} \stackrel{n-1}{\longleftrightarrow} t_{-2} \stackrel{n-2}{\longleftrightarrow} \cdots \stackrel{1}{\longleftrightarrow} t_{-n}.\]
    \begin{lemma}
      Transformation formulas among the $r_j^{(k)}$, $r_{\ol{j}}^{(k)}$, and $h_k$ for varying $k$.
    \end{lemma}
    \begin{proof}
      For $i<n-k+1$, we have
      \[\Delta_{\omega_i}(g_k)=h_k^{\omega_i}.\]
      For $i=n-k+1$, we have
      \[\Delta_{\omega_{n-k+1}}(g_k)=h_k^{\omega_{n-k+1}}\big(1+r_{n-k+1}^{(k)}r_{\ol{n-k+1}}^{(k)}\big).\]
      For $i>n-k+1$ we have
      \[\Delta_{\omega_i}(g_k)=h_k^{\omega_i}\big(1+r_i^{(k)}r_{\ol{i}}^{(k)}(\cdots)\big).\]
      The $(\cdots)$ depends on the root system, I'm not sure the best way to write it down.
    \end{proof}

    \begin{lemma}
      $\hat y_i^{(0)}=r_i^{(0)} r_{\ol{i}}^{(0)}$\\
      $\hat y_i^{(k)}=r_i^{(k)} r_{\ol{i}}^{(k)}$??\\
    \end{lemma}
    \begin{theorem}
      Every generalized minor is pointed on $G^{c,c^{-1}}$.
    \end{theorem}
    \begin{proof}
      Refactor the generic element 
      \[x_{\ol{1}}(t_{\ol{1}})\cdots x_{\ol{n}}(t_{\ol{n}})h x_n(t_n)\cdots x_1(t_1)\]
      as
      \[x_{\ol{1}}(t_{\ol{1}})\cdots x_{\ol{n}}(t_{\ol{n}}) x_n(t_nh^{?})\cdots x_1(t_1h^{?}) h.\]
      Then it is essentially obvious by thinking about walking the representation.
    \end{proof}

  \begin{thebibliography}{XXX}
    \bibitem[Qin]{qin} Fan Qin, ``Bases for upper cluster algebras and tropical points''
    \bibitem[RSW19]{rsw19} Dylan Rupel, Salvatore Stella, Harold Williams: ``Affine cluster monomials are generalized minors.'' Compositio Math. 155 (2019) 1301-1326. 
  \end{thebibliography}
\end{document}
